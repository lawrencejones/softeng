%----------------------------------------------------------------------------------------
%   PACKAGES AND OTHER DOCUMENT CONFIGURATIONS
%----------------------------------------------------------------------------------------

\documentclass[11pt]{article}

\usepackage[english]{babel}
\usepackage[utf8x]{inputenc}
\usepackage{amsmath}
\usepackage{graphicx}
\usepackage{csquotes}
\usepackage{hyperref}
\usepackage{fancyvrb}
\usepackage{url}
\usepackage[a4paper, total={6in, 8in}]{geometry}
\usepackage[colorinlistoftodos]{todonotes}

\setlength{\parskip}{0.4em}

%----------------------------------------------------------------------------------------
%   HEADING
%----------------------------------------------------------------------------------------

\newcommand{\BigO}[1]{\ensuremath{\operatorname{O}\left(#1\right)}}

\title{\textsc{Software Engineering}\\Static Analysis}
\author{Lawrence Jones \{lmj112\} \  Alice Sibold \{as4712\} \\
        Joshua Coutinho \{jrc12\}}

\date{}
\begin{document}
\maketitle

%----------------------------------------------------------------------------------------
%   BODY
%----------------------------------------------------------------------------------------

\textit{Git repository to be found at
\path{https://gitlab.doc.ic.ac.uk/as4712/softeng-sa}}

\subsection*{Question: How can static analysis techniques be most effectively
integrated in a continuous delivery development process?}

Static analysis is the process of analysing source code, or a binary, for
potential issues, \textit{without executing the program}. This is a proven
technique which, despite some limitations\cite{sethi15}\cite{anderson08}, is
actively used in production. Once such tool is
\href{http://fbinfer.com/}{INFER}, which is being used internally at Facebook.

Using static analysis tools in a continuous integreation environment has two
key challenges\cite{calcagno15}; the first is developer confidence. Facebook
calls this the \textit{"Social challenge"}, where developers need to trust that
the tool isn't reporting bugs that aren't really there. False positives ensure
users tune out the results and reduce confidence in the tooling, meaning true
issues are likely to be ignored.

Conversely, if the analysis regularly reports \textbf{false negatives}, i.e.
misses real bugs in the program, developers will lose faith in the value of the
tool, as problems must be found manually anyway.

Tuning the sensitivity for the right balance of false positive/negatives, so
that most true bugs are found within the program while keeping a low level of
false negative noise is crucial, and a task that can benefit from constant
attention. Having an internal development team responsible for static analysis
tooling can help achieve this balance\cite{calcagno15}, as well as providing a
communication point for any concerns developers might have.

The second concern is to preserve the developers' workflow as much as possible,
making the tooling a natural extension of developer habits instead on an
intrusive extra burden.

Facebook did this by integrating the verification stage into the code review
process, whereby any source code diffs up for peer review must pass the static
analysis, on top of unit tests. This is made possible by aggressively caching
analysis results to enable incremental scans, reducing the time to run the tools
to about 10m for each new change, a time that Facebook report is interactive
enough to be useful to developers in their workflow.

Nightly full non-cached analysis is performed to check for any bugs that might
have been missed, striking a balance between providing value in a lightweight
form for developers as they make changes, while ensuring complete coverage of
the code.

%----------------------------------------------------------------------------------------
%BIBLIOGRAPHY
%----------------------------------------------------------------------------------------

\begin{thebibliography}{99}

\bibitem{sethi15} \emph{Understanding Strengths and Limitations of Static
  Analysis Security Testing (SAST)} \\
  \path{https://labs.securitycompass.com/appsec-2/understanding-strengths-and-limitations-of-static-analysis-security-testing-sast/}
\bibitem{anderson08} \emph{The Use and Limitations of Static-Analysis Tools to
  Improve Software Quality} \\
  \path{https://buildsecurityin.us-cert.gov/resources/crosstalk-series/the-use-and-limitations-of-static-analysis-tools-to-improve-software-quality}
\bibitem{calcagno15} \emph{Moving Fast with Software Verification} \\
  \path{https://
  research.facebook.com/publications/422671501231772/moving-fast-with-software-verification/}
  \end{thebibliography}

\end{document}

