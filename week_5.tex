%----------------------------------------------------------------------------------------
%   PACKAGES AND OTHER DOCUMENT CONFIGURATIONS
%----------------------------------------------------------------------------------------

\documentclass[11pt]{article}

\usepackage[english]{babel}
\usepackage[utf8x]{inputenc}
\usepackage{amsmath}
\usepackage{graphicx}
\usepackage{csquotes}
\usepackage{hyperref}
\usepackage{fancyvrb}
\usepackage{url}
\usepackage[a4paper, total={6in, 8in}]{geometry}
\usepackage[colorinlistoftodos]{todonotes}

\setlength{\parskip}{0.4em}

%----------------------------------------------------------------------------------------
%   HEADING
%----------------------------------------------------------------------------------------

\newcommand{\BigO}[1]{\ensuremath{\operatorname{O}\left(#1\right)}}

\title{\textsc{Software Engineering}\\Week V~- Continuous Delivery}
\author{Lawrence Jones \{lmj112\} \  Alice Sibold \{as4712\} \\
        Joshua Coutinho \{jrc12\}}

\date{}
\begin{document}
\maketitle

%----------------------------------------------------------------------------------------
%   BODY
%----------------------------------------------------------------------------------------

\textit{Git repository to be found at
\path{https://github.com/lawrencejones/softeng-ci} }

\subsection*{Question: What are the problems that a build pipeline is trying to
solve?}

Complex applications are typically deployed to a number of different
environments. Without an automated build pipeline, these deployments are a
manual process subject to human error and are difficult to repeat reliably. Any
change in the deployment process will require communicating to the rest of the
team, and it's likely that changes made will lack documentation.\cite{bsdp}.
Relying on developers to manually verify their code can lead to each developer
verifying different things, and it becomes difficult to enforce static analysis
and testing that a team would ideally have run on each change to their project,
in each of the environments it will be deployed to.\cite{methodsntools}

Build pipelines can remove the uncertainty from deployment while providing rapid
feedback to each change a developer makes. By enforcing that each change passes
the CI steps before being pushed into the main project branch, we can avoid
monolithic commits that are symptomatic of an unstructured build process,
avoiding the process loss they entail with several back and forths between QA
and development\cite{cimartinfowler}.

Having an entire team buy into a build pipeline ensures that everyone can
benefit from maintaining the automated procedure, removing the need for each
developer to relearn the process each change. Spending the time to ensure the
automated process is solid- with parallel build steps and good coverage- then
becomes worthwhile.

Ideally a good pipeline will deploy to each environment as part of the standard
build. This removes a large class of errors where code will work on a developers
machine but break in production, adding confidence that deploying this change
won't cause issues. Testing code with a view for compatibility can be difficult
and laborious when a manual process, while pipelines can facilitate a clear
strategy for testing code against each target platform.

These characteristics of a pipeline provide a system with increased robustness
to change, and see an increase in developer productivity by tightening the
feedback loop between making a change and seeing its effects\cite{whitepaper}.

%----------------------------------------------------------------------------------------
%BIBLIOGRAPHY
%----------------------------------------------------------------------------------------

\begin{thebibliography}{99}

\bibitem{whitepaper} \emph{The Deployment Production Line} \\
	\path{http://continuousdelivery.com/wp-content/uploads/2011/04/deployment_production_line.pdf}
\bibitem{cimartinfowler} \emph{On Continuous Integration, Martin Fowler}
	\path{http://www.martinfowler.com/articles/continuousIntegration.html}
\bibitem{bsdp}\emph{Building a Software Deployment Pipeline}
	\path{http://lethain.com/building-a-software-deployment-pipeline/}
\bibitem{methodsntools} \emph{Continuous Delivery using Build Pipelines}
\path{http://www.methodsandtools.com/archive/archive.php?id=121}
\end{thebibliography}

\end{document}
