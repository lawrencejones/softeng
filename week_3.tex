%----------------------------------------------------------------------------------------
%   PACKAGES AND OTHER DOCUMENT CONFIGURATIONS
%----------------------------------------------------------------------------------------

\documentclass[11pt]{article}

\usepackage[english]{babel}
\usepackage[utf8x]{inputenc}
\usepackage{amsmath}
\usepackage{graphicx}
\usepackage{csquotes}
\usepackage{fancyvrb}
\usepackage[a4paper, total={6in, 8in}]{geometry}
\usepackage[colorinlistoftodos]{todonotes}

\setlength{\parskip}{0.4em}

%----------------------------------------------------------------------------------------
%   HEADING
%----------------------------------------------------------------------------------------

\newcommand{\BigO}[1]{\ensuremath{\operatorname{O}\left(#1\right)}}

\title{\textsc{Software Engineering}\\Software Evolution}
\author{Lawrence Jones \{lmj112\} \  Alice Sibold \{as4712\} \\
        Joshua Coutinho \{jrc12\}}

\date{}
\begin{document}
\maketitle

%----------------------------------------------------------------------------------------
%   BODY
%----------------------------------------------------------------------------------------

\subsection*{Can you tell from the version control history whether a system is
well architected?}

To answer this question, we first need to define the concept of software
architecture, and what it means to us as developers. The book `Software
Architecture in Practice'~\cite{softwareArchitectureInPractice} defines
architecture as\dots

\begin{displayquote}
The software architecture of a program or computing system is the structure or
structures of the system, which comprise software elements, the externally
visible properties of those elements, and the relationships among them.
\end{displayquote}

This definition suggests that architecture is an abstraction, a reduction of the
system as a whole. The external elements in the case of our software become the
interactions between the abstracted modules used to compose our projects,
reduced from the minutiae of each line of code into the overall shape of the
structure.

For developers, architecture is essential to enable developers to communicate
effectively about complex systems~\cite{anIntroductionToSoftwareArchitecture}.
It is a reduction from the gritty details of the code into a general shape that
each developer can retain in their heads, making use of our human ability to
apply patterns to simplify problems. The abstraction it defines also specifies
boundaries that aim to make the system robust to modification by isolating the
noise of each change to minimise disruption to unrelated code.

As an architecture can never be perfect, we must judge its quality on whether
it optimises for the majority of changes made to our codebase. Adam Tornhill
proposes we focus our attention on `hotspots'~\cite{codeAsCrimeScene} in our
code, places where activity is highest and as a result represent the general
case of work done on our system. Hotspots are classified by both the churn of
the file (how often it is modified) and the number of authors who work on it.

\begin{table}
  \begin{tabular}{ | l | l | l | }
    \hline
    \textbf{Entity} & \textbf{N Authors} & \textbf{N Revisions} \\
    \hline \hline
    \texttt{lib/model.js} & 5 & 82 \\ \hline
    \texttt{lib/document.js} & 4 & 68 \\ \hline
    \texttt{lib/query.js} & 4 & 47 \\ \hline
    \texttt{lib/utils.js} & 3 & 26 \\ \hline
    \texttt{lib/connection.js} & 2 & 27 \\ \hline
    \texttt{lib/drivers/node-mongodb-native/connection.js} & 3 & 17 \\
    \hline
  \end{tabular}
  \caption{Top hotspots for \texttt{Automattic/mongoose}~\cite{mongoose}}
  \label{table:mongooseHotspots}
\end{table}

Table~\ref{table:mongooseHotspots} is an example of the results you can get from
analysing repositories for hotspots. We can verify our architecture by searching
for the commits that touch these hotspots and checking if the change incurred a
knock on effect to architecturally distinct modules. This pattern of change is
called temporal coupling, where files change in lockstep, and provide a window
into our architectural integrity via our VCS history.

From the temporal coupling of these hotspots, we can estimate the cost
associated with the majority of changes that are committed to the codebase.  If
we find that files \texttt{a.js}, \texttt{b.js} and \texttt{c.js} are highly
coupled to \texttt{lib/model.js}, then \texttt{mongoose}'s architecture is
failing to isolate disruption from the typical change, and each of these
hotspots is generating peripheral activity that increases the system fragility.

We believe that VCS history can not only unearth important details about our
applications architecture, it can also help us escape our human biases and
remove some of the subjectivity behind it. Martin Fowler states that
`architecture boils down to the important stuff— whatever that
is'~\cite{patternsOfEnterpriseApplicationArchitecture}. Our VCS history provides
an objective source of information that we can use to clarify discrepancies in
developer understanding of how our systems work, and in doing so reduce the
communication issues that arise from those differences.

%----------------------------------------------------------------------------------------
%   BIBLIOGRAPHY
%----------------------------------------------------------------------------------------

\medskip
\bibliography{lib/refs}
\bibliographystyle{unsrt}
\newpage

\end{document}
